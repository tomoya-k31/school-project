\documentclass[a4j]{jarticle}
\usepackage{graphicx}
\usepackage{verbatim}
\usepackage{ascmac}
\usepackage{url}
\usepackage{listings,jlistings}
\usepackage{color}

\input{/home/ryousuke/listings_temp.tex}

\title{情報科学プロジェクト実験レポート課題}
\author{S142063 佐藤涼亮}

\begin{document}
\maketitle
\begin{center}
{\LARGE \underline{Webクライアント}}
\end{center}
\section{課題の内容}
{\large \underline{Webページの取得}}
\subsection{要点}
\begin{itemize}
\item Webページを取得するプログラムの作成。(htmlの内容)
\item 配布されたプログラム(cil.cpp)をもとに作成。
\item 対象となるWebページは、そのURLをコマンド引数として指定する。
\item 取得するWebページのサイズは予めわからないため、サーバが接続を切るまで繰り返し受信を行う。
\end{itemize}

\section{プログラムの説明}
実行時、コマンドライン引数に取得するWebページのURL(例:\url{http://www.cc.seikei.ac.jp/index.html})を指定する。
プログラム上でURLを受け取り、ホスト部(例:www.cc.seikei.ac.jp)とパス(例:/index.html)に分解し入手する。
入手したホスト部の情報をもとに接続先情報の準備を行う。(ポート番号は80で固定)
その情報をもとにTCPソケットの作成を行う。
作成したソケットを用いて準備したホストへの接続を試みる。
接続が成功すれば、HTTPリクエストのGETの要求を入手したパスで指定してHTTP/1.0でサーバに送信する。
メッセージバッファサイズは128で固定し、たりない分は繰り返し終わるまで受信を繰り返す。
表示する部分は、HTTPリクエストの部分を省いた、空行以降(htmlの内容)を表示する。
最後まで表示したら、サーバとの接続を遮断し終了する。

\subsection{目的}
TCP/IP を利用した通信プログラムの作成

\subsection{方法}
HTTPによるWebサーバへの要求(HTTPリクエスト)

\subsection{結果}
\url{http://islay.ci.seikei.ac.jp/} と \url{http://islay.ci.seikei.ac.jp/~okam/index.html} を引数にした時の実行結果を以下に記す

\lstinputlisting[caption=islay.ci.seikei.ac.jp,label=islay.ci.seikei.ac.jp,language=HTML5]{islay.txt}

\lstinputlisting[caption=islay.ci.seikei.ac.jp,label=islay.ci.seikei.ac.jp,language=HTML5]{okam.txt}

\subsection{考察}
HTTPリクエスト部分の表示を省き、htmlの内容のみの出力に成功し、期待通りの結果が得られた。
\newpage
\section{感想}
今まで、授業でもC++のプログラムは学んできたが、オフラインで行うことしかなかった。\\
しかし、今回の課題ではWeb上のページにアクセスして、Webページを取得するプログラムを作成した。
より高度なプログラムに触れて、いっそうプログラミングへの意識が高まった。
このような、プログラムを作ることで、以前に他の授業で学んだソケットの話や、クライアントとサーバ間の通信の動きについてより理解を深められた。

「プログラミングの理解を深めるならC言語やC++を勉強すると良い」という話を聞いた事がある。
C/C++の言語は、プログラミング言語の中でも複雑ではあるが、基礎となっていることが多いと言われている。
そのため、この言語をより学ぶことでプログラミングの理解を深め、他言語でも活かせるようなプログラマーを目指したい。


\section{プログラム}
\lstinputlisting[caption=report.cpp,label=report.cpp]{report.cpp}

\end{document}
